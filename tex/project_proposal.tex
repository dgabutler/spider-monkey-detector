\documentclass[11pt]{article}

\usepackage[a4paper, margin=2cm]{geometry} % page size
\usepackage[utf8]{inputenc} 

%% Font
\usepackage{helvet}
\renewcommand{\familydefault}{\sfdefault}

%% References
\usepackage{natbib}
\bibliographystyle{agsm}
\usepackage{url}

%%	Table
\usepackage{booktabs}
\usepackage[table,xcdraw]{xcolor}
\usepackage{caption}
\captionsetup{labelsep=space,justification=justified,singlelinecheck=off}

%%	Graphics
\usepackage{graphicx} 

%%	Double spacing
\usepackage{setspace}

%%	Line number
\usepackage{lineno}

%%	Section spacing
\usepackage{titlesec}
\titlespacing*{\section}{0pt}{15pt plus 0pt minus 0pt}{5pt plus 0pt minus 0pt}

\begin{document}
\begin{titlepage}
	
	\newcommand{\HRule}{\rule{\linewidth}{0.5mm}} % Defines a new command for the horizontal lines, change thickness here
	
	\includegraphics[width=8cm]{../Sandbox/Figures/imperial_logo.png}\\[1cm]
	
	\center % centre all on page
	
	\LARGE MSc PROJECT PROPOSAL\\
	\textsc{\large 8th April 2018}\\[1cm]
	\textsc{\Large Computational Methods in Ecology and Evolution}\\[0.5cm]
	\textsc{\large Department of Life Sciences}\\[0.5cm]
	
	\HRule \\[0.4cm]
	{ \huge \bfseries Creating a basic automated detection system for the calls of Geoffroy's spider monkey, \textit{Ateles geoffroyi}, and assessing level of accuracy possible with relatively small training database}\\[0.2cm] 
	\HRule \\[1.5cm]
	
	\begin{minipage}{0.4\textwidth}
		\begin{flushleft} \large
			\emph{Author:}\\
			Duncan Butler\textsuperscript{1}
		\end{flushleft}
	\end{minipage}
	~
	\begin{minipage}{0.4\textwidth}
		\begin{flushright} \large
			\emph{Supervisors:} \\
			Jenna Griffiths\textsuperscript{2} \\
			James Rosindell\textsuperscript{3} \\[1.2em] 
		\end{flushright}
	\end{minipage}\\[1.5cm]
	~
	\begin{minipage}{0.85\textwidth}
		\begin{flushleft} \small
			(1) Department of Life Sciences, Imperial College London, Silwood Park Campus, Ascot, Berkshire SL5 7PY, UK. {duncan.butler17@imperial.ac.uk}\\[0.5em] 
			(2) Department of Life Sciences, Imperial College London, Silwood Park Campus, Ascot, Berkshire SL5 7PY, UK. {j.griffiths17@imperial.ac.uk}\\[0.5em] 
			(3) Department of Life Sciences, Imperial College London, Silwood Park Campus, Ascot, Berkshire SL5 7PY, UK. {j.rosindell@imperial.ac.uk} \\[0.5em] 
			
		\end{flushleft}
	\end{minipage}\\[2em]
	%\makeatother % change @ back
	
	\vfill % Fill the rest of the page with whitespace
	
	\end{titlepage}

\linenumbers
\onehalfspacing

\section{Keywords}
acoustic monitoring, speech processing, primatology, convolutional neural networks, deep learning, feature extraction

\section{Introduction}

A huge amount of ecological information, inavluable for conservation work, is conveyed in audio data in rainforests. With recent advances in technology this can now be collected comparitively cheaply in a process known as passive acoustic monitoring (PAM). This audio data then needs to be analysed for the presence of sounds of interest (detection), such as animal calls, and the allocation of these sounds to, for example, a given species (classification). The benefits of PAM, outlined in \cite{heinicke2015assessing}, include an increased ability to monitor in dense rainforest where detecting species visually can be challenging, and a significant reduction in levels of human disturbance caused. 

In recent years, PAM analyses have benefitted greatly from the application of machine learning to automate the detection and classification process. This offers significant advantages such as the ability to increase monitoring spatial scale for comparitively little extra cost, a significant saving of time and money as an expert is not required to listen to all of the recorded hours \citep{heinicke2015assessing}, and removal of observer bias inherent in other monitoring approaches \citep{wrege2017acoustic}. Machine learning algorithms compare key features - temporal (e.g. duration) or spectral (e.g. peak frequency) characteristics - of detected sounds with those in a learned sound library created using a training data set, and return the most likely match via the use of probabilistic scores \citep{reason2016recommendations}. While these key features of sounds of choice have typically been chosen manually, a limiting factor in applying these techniques to highly biodiverse regions is that it is difficult to extract features sufficiently different to reliably classify audio data. This is due to animals often calling at similar frequencies \citep{slabbekoorn2004habitat} and large amounts of ambient noise of very variable intensity \citep{waser1977experimental}. As a result of this, currently-existing libraries are biased toward temperate regions, with work to correct for these biases being described as a major gap in the field \citep{browning2017passive}. 

Recent theoretical advances, for example in deep convolutional neural networks (CNNS) \citep{lecun2015deep,goeau:hal-01373779} - notable for outperforming other algorithmic methods and even human experts \citep[e.g.][]{kiskin2017mosquito} - have made it possible for algorithms to automatically learn which features of the data are able to best classify recorded sounds. As this can avoid the process of dimensionality-reduction (and therefore information loss) preceding selection of hand-designed features, this technique has been shown to substantially increase the accuracy and robustness of automated detection systems \citep{stowell2014automatic,browning2017passive}. These emerging methods have already been applied to acoustic monitoring \citep[e.g.][]{goeau:hal-01373779,macaodha2018bat} and have been predicted to become much more widely used \citep{browning2017passive}.

This project would apply these techniques – testing the accuracy of around 10-20 current machine learning algorithms, using learned features (e.g. by deep CNNs) as well as manually labelled features - in an attempt to develop a basic automated detection system for the calls of the endangered neotropical primate Geoffroy’s spider monkey, \textit{Ateles geoffroyi}. For several reasons – being almost entirely arboreal, frugivorous (patchily-distributed food), with complex fission-fusion societies (splitting into subgroups to forage) – spider monkeys are heavily reliant on acoustic communication and have a repertoire of around 10 different calls of known meaning \citep{ramos2008communication} conveying a great deal of information. The detection system will enable the collection of a large amount of data, allowing for an increase in our current understanding of the ecology (behaviour, distribution and use of territory) of populations on the Osa peninsula of Costa Rica. The work in this project would fit into a larger effort to design wildlife corridors suitable for \textit{A. geoffroyi} - as habitat of suitable for this target species is known to then be of sufficient quality for a number of other threatened species - to connect the populations currently isolated on the peninsula with unnocupied suitable habitat further inland. 


\section{Methods}

Using a small starter labelled training database provided, carry out process of feature selection and extraction (exploring hand-designed features but concentrating on learned feature approaches such as neural networks). Using R (e.g. 'seewave') and/or Python (e.g. pyAudio), test the ability and accuracy of 10-20 machine learning algorithms to automatically detect and classify \textit{A. geoffroyi} calls. Accuracy will be increased by feeding identified calls back into the algorithms as training data, and algorithms will be tested using data collected during one month of fieldwork. 


\section{Outcomes}
An expanded database of labelled \textit{A. geoffroyi} calls and labelled ambient sounds, from an environment currently underrepresented in the literature \citep{browning2017passive}. The main outcome will be a basic automated detection and classification system for spider monkey calls (which will be built on as data is collected and labelled in upcoming years on this overall project).

\section{Project Feasibility}
\begin{table} [hbtp]
	\caption{- Gantt chart. Final week left empty for minor adjustments}
	\label{tab:gantt}
	\includegraphics[width=\textwidth,height=\textheight,keepaspectratio]{../Sandbox/Figures/gantt-crop}
\end{table}

\section{Budget}
\begin{itemize}
	\item $\pounds$590: Flights to Costa Rica
	\item $\pounds$550 ($\pounds$50, $\pounds$300, $\pounds$200) : In-country transport, accomodation, and food
\end{itemize}

\newpage
\bibliography{project_proposal}
	
\end{document}
